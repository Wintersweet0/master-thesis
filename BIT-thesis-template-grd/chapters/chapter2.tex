%%==================================================
%% chapter02.tex for BIT Master Thesis
%%==================================================
\chapter{相关工作}
本章对本文研究内容的相关工作进行简要介绍。首先是基于以太坊的树状区块链;第二是GeoHash地理信息,第三是出租车调度系统的发展现状,第四是导航算法(简单描述)

其一是国内外现有路况计算算法,其二是分布式计算领域的计算框架和算法设计

2.1 路况计算
2.1.1 基于浮动车轨迹的路况计算算法
两篇基于瞬时速度的文献,瞬时速度不一定是可靠的,比如路口拥堵或红绿灯,而且采样率低。
低采样率:两篇文献采用平均速度,一篇采用历史道路速度加权,缺点在于不考虑路口,且轨迹点间可能包含多条道路。
一篇文献考虑了路口的影响,但是不能估计车辆的路口转向时间,且路况时效性不高。
关于路口转向,[27][28]。[11]将一部分道路路况分离到路口,张禹[17]根据历史路况得出道路速度和估计路口转向时间,然后将每辆车的真实时间加权分配。缺点是没有对立交桥进行区分处理。

本文在张禹[17]的基础上将立交桥纳入考虑范围。

2.1.2 路口的定位与范围划分
[15][26]将路口范围设置为统一值。[11]假设两个轨迹点间只能跨越一个路口,局限性大。[17]以三元组描述路口转向时间,解决两点之间跨越多个路口的情况。
[29][30]对立交桥进行几何学提取,缺点是需要依赖遥感和图形库。[31]通过聚类对复杂交叉口进行定位,但没有应用于城市整体。本文将立交桥抽象为路口,利用密度聚类对立交桥进行自动化定位。[31]将立交桥划分为圆形,不适合形状不规则的立交桥。本文使用道路连接点构成路口,将立交桥和路口一并存入数据库考虑。

2.2 分布式计算
2.2.1 分布式框架选择
不同分布式框架的数据处理机制不同,但覆盖了计算模式[32]
1. Hadoop
数据访问延迟比较高,主要工作集中在对实时性要求不高的离线批次处理[33,34],[35]搭建对海量出租车数据进行预处理,[18]离线处理交通流量。但MapReduce开发成本高效率低,不适合实时路况分析。
2. Storm
storm采用全内存计算,拥有更低的处理延时,弥补hadoop实时性方面的缺陷,但是吞吐量低于hadoop。
storm已经可以满足对实时性和容错性的要求,[36]在storm预测交通流量。storm存在数据成倍增加时吞吐量的瓶颈问题。[20]将Hadoop与storm结合对离线数据和实时路况进行分析。
基于storm的交通路况分析研究成果较少。
3.Spark
spark由加州大学伯克利分校的 AMP 实验室基于与 Hadoop 框架下 MapReduce 编程范式相同的原则开发而成[37]。[38]介绍spark的框架与处理机制。运行速度是hadoop的百倍以上。spark的RDD有很强的容错能力。
Spark Streaming可以满足除了对实时性有严苛要求以外的绝大多数实时流处理场景。其吞吐量要比 Storm 高出 2~5 倍[39],在节点计算出现故障时重演时间的代价相对较小[40]

目前已有很多学者,文献[41]结合 Spark 框架进行的动态城市路况导航,文献[3]中采用 Spark 处理出租车轨迹数 据进行乘客出行距离分布特征、出租车使用时间分布特征、出租车需求特征等一系列分析, 文献[42]基于 Spark 和出租车的历史轨迹数据构建平均速度概率分布模型进行实时路况预测等。在路况计算方面采用 Spark Streaming 计算框架的研究虽鲜有耳闻,但基于 Spark Streaming 框架进行的实时数据分析研究[43]证实了使用该框架对海量网络数据实时分析的可行性。

表 2-1 对三种分布式计算框架的部分特性进行了对比。Spark 秒级的延时处理和吞吐量大的特点更加适用于计算环境。因此本文决定选 择综合性能更佳的 Spark 计算框架来实现分布式路况计算系统。



2.2.2 分布式算法设计
文献[44]以车辆为载体,通过分布式哈希表将采集到的交通数据流分散到多个处理节点完成。文献[45]对地图区域进行了划分,每个计算节点只存储部分区域的地图。但文献[45]的缺点是当车辆跨越地图区域时,会涉及到区域间的路况交互和车辆的 GPS 历史记录迁移。
本文提出一种按车辆划分的方法,不同 Executor 分别处理不同车辆编号的轨迹,以此加快计算速度,并通过划分合适的批处理时间间隔(Batch Interval),以保证路况计算的时效性。其优点是避免了对地图数据进行切割,且由于每个 Executor 相互独立,使得系统资源具有良好的可伸缩性,即无论增加还是减少 Executor, 均不影响分布式路况计算系统的计算结果,因此本文还引入了自适应的动态资源 (Executor)增减机制,根据数据量的变化动态调整 Executor 数量,并使处理时间保持在基本一致的状态。缺点是各个车辆计算出的路况信息进行合并时,存在一定的合并开销。

2.3 本章小结
本章首先对国内外路况计算领域的相关算法进行了介绍,基于车辆行驶速度的路况计算是目前主流的路况计算方法,但相关研究[12]表明车辆在路口消耗的时间占全部行驶时间 的 25 \% 以上,因此还需要考虑到路口转向时间,当前路口转向时间对于立交桥这种复杂的大型道路交叉口鲜有考虑,因此有必要设计一种兼顾普通路口和立交桥的路况计算优化算法。
对 Hadoop、Storm、Spark 框架进行了介绍和对比后,决定选用基于内存计算的 Spark 框架来进行分布式路况计算系统的开发,并简要介绍了路况计算领域相关的分布式算法设计。



\section{基于以太坊的树状区块链}
介绍区块链的性能瓶颈,引出树状区块链的研究目的、简要介绍原理。
\subsection{传统区块链的特点和不足}
\subsection{树状区块链的特性}

\section{GeoHash地理信息}
由于传统计算两个经纬度所表示坐标点距离时需要使用球面距离公式,若在以 GeoHash 为坐标表示的系统中沿用这套算法,则丧失了 GeoHash 带来的计算简便性。利用GeoHash编码的特点进行距离计算,避免了复杂的三角函数和球面计算,并且适用于对小数支持较弱、不提供复杂数学函数计算支持的区块链智能合约编写语言 Solidity。
\subsection{基于GeoHash的地图}
将矢量地图数据以GeoHash形式存到区块链上的原理、优点分析和改进空间。
\subsection{GeoHash几何计算原理}
介绍GeoHash数格子进行几何距离计算的原理。
\subsection{leaflet矢量地图渲染框架}
介绍leaflet矢量地图渲染框架,以及其对GeoHash数据的支持情况。

\section{出租车调度系统}
介绍出租车调度系统的研究现状,指出中心化系统的不足,容易引发信息垄断、信息泄露和恶性竞争。
\section{导航算法}
介绍导航算法的发展历史和研究现状,分类简要介绍几种导航算法的优缺点,指出区块链应用缺乏导航算法的支持,以及在区块链中将导航算法与GeoHash结合的优点。
