%%==================================================
%% chapter02.tex for BIT Master Thesis
%%==================================================
\chapter{相关工作}
本章对本文研究内容的相关工作进行简要介绍。首先是基于以太坊的树状区块链,具体涉及传统区块链的性能瓶颈和树状区块链的结构特性。第二是GeoHash地理信息,具体介绍GeoHash地理信息便于区域绑定和查询的特点,基于GeoHash的几何原理,以及矢量地图渲染框架leaflet。第三是出租车调度系统的发展现状,并指出其安全性上的不足和解决方法。第四是导航算法,介绍导航算法的发展种类,对比其性能和特点,在此基础上简要介绍基于GeoHash的导航算法设计。\par

\section{基于以太坊的树状区块链}
介绍区块链的性能瓶颈,引出树状区块链的研究目的、简要介绍原理。简单参考董斌的论文。

树状区块链融合了地理信息,分为创世块、分支块(只维护直接下层区域的索引信息)和普通块(记录在对应地理区域内发生的交易)。

区块链是一个共享的不可更改的总账,它用于记录交易、跟踪资产以及建立信任\upcite{王震2019面向大数据应用的区块链解决方案综述}。几乎所有有价值的东西都可以在区块链网络上进行跟踪和交易,从而降低了风险并削减了所有相关成本。区块链是传递交易信息的理想选择,因为它可以提供在不可更改的总账中的即时、共享且完全透明的信息,这类信息同时具备一定的安全性,只能由获得许可的网络成员访问。另外,区块链网络也可以跟踪订单、付款、账户、生产等等,因此可以通过此类技术手段减少不必要的交易纠纷。\par
智能合约是存储在区块链上的程序,可以在满足预定条件的情况下运行,它们通常是自动执行的脚本,以便所有参与者无需任何中介机构的参与就可以立即结果,极大保证安全性。智能合约代码语句十分简单,当预定条件已经得到满足并完成验证时,区块链网络将执行对应动作,比如释放资金,发出凭证等,然后交易完成时更新区块链。由智能合约完成的交易也是无法更改的,只有获得许可的参与者才能看到结果。本文是基于以太坊改进的树状区块链平台来设计和部署智能合约,在此基础上进行出租车调度系统的研究。\par

\subsection{传统区块链的特点和不足}
传统区块链的发展,发展到以太坊,以太坊性能上的不足,以及解决以太坊性能的方法。
1. 安全性和性能上的不足
2. 对地理信息缺乏自有支持
3. 要满足车载自组网的灵活和移动特性

利用智能合约设计系统

传统区块链技术基于区块,整个网络同时只有一条单链,基于 PoW 共识机制出块无法并发执行,无法满足车载自组网对高并发操作和网络稳定性的需求,传统区块链也不具备移动性,无法与地理位置信息进行绑定,不能有效利用车载网中地理信息的区域化特性。另外,传统区块链的单链结构,要求所有节点必须在同一区块链中,这将会导致节点数目和数据量过大,不满足车载自组网中车辆节点的移动特性,且一旦出现网络分区,就会对整个区块链产生很大影响。目前采用分片技术更改原始单链的链式结构是解决上述问题的主流方法\upcite{2018Blockclique}。当前也有多链结构的相关工作。\par
根据多链结构是否改变底层区块链结构可分为应用多链和结构多链。应用多链,即在应用层面建立不同功能的多个区块链,没有修改底层区块链结构。结构多链,即根据需求对区块链底层结构进行调整的多链结构。\par
应用多链
Shrestha 等\upcite{2019Regional}研究了区域区块链在车载自组网中的应用。由物理边界区域内的节点共享的区块链,区域内部的传播延迟比较小,可以极大减少消息延迟,但此研究并没有设计跨区域和跨链交易的相关内容。\par
Hirtan 等\upcite{2019Blockchain}提出了一种包含专用链和公用链的医疗保健系统。专用链保存患者的真实身份信息;公用链存储患者的健康信息以及临时 ID 数据,实现隐私数据保护和可用数据公开访问。此研究通过特定节点掌握患者临时 ID 和真实身份的关联,来实现两个区块链数据的传递。\par

结构多链
Youngjune 等人\upcite{2019Monoxide}建立多个独立并行的单链共识组。各个共识组地位平等,大部分交易只在组内完成,跨组交易则采用异步方式将中继事务发送到目标区域,而不是整个网络,减轻了网络负载。值得一提的是,为了确保每个区域中的有效采矿能力与整个网络处于同一水平,采用了诸葛弩改进 PoW 的挖矿方式,这也同时保证了分组抵御攻击的能力。但其共识组分区方法没有考虑实际地理位置信息,同时车载自组网中跨区域交易的规模较大,此研究的跨组交易的网络开销较大,不适用于车载自组网。\par
Zamani 等人\upcite{2018RapidChain}提出了基于分片的公共区块链协议,将节点划分为多个较小的称为委员会的节点组,节点组在不相交的交易块上并行操作,并维护不相交的独立账本,也就是分片,分片由每个成员以区块链的形式存储。为了解决节点频繁移入移出对网络造成的影响,将委员会分为活跃和不活跃两个分类,节点创建后,第一次进入委员会需要加入活跃类,再次进入或转移时需要加入不活跃类。但委员会内节点数量固定,缺少灵活性;委员会构建和重构时不涉及地理因素,增加了更新时间,不适用于车载自组网。\par

Byung 等人\upcite{Byung2018Hybrid}将物联网与基于区块链的智能合约相结合,用于结构健康监视 (SHM)。这个区块链物联网网络分为核心和边缘网络。边缘节点充当查询实时响应的集中式服务器,并提供低延迟和带宽使用率,核心网络由具有高存储容量的miner 节点组成,负责生成新块、验证工作证明,并包含自主决策的智能合约,这种划分提高了系统的效率和可伸缩性,但核心网络十分庞大是主要问题。核心网络的庞大影响移动节点的交易效率。\par

Pajooh 等人\upcite{2021Multi}提出了一种多层区块链安全模型来保护物联网网络,同时利用群集的概念来简化多层架构,通过使用模拟退火和遗传算法相结合的混合进化算法来定义物联网 K- 未知簇,选择的群集头负责本地身份验证和授权,增强了网络认证机制的安全性,显示出更适合的平衡网络延迟和吞吐量。但上述两类区块链研究没有考虑到地理因素,跟适用于车载自组网的区块链结构还有一定的距离。\par

Ochôa 等人\upcite{2019A}提出了侧链结构,由 BlockPRI、BlockSEC、BlockTST 三个不同的区块链,使用了三个区块链来确保系统的隐私性,安全性和信任性。BlockPRI 存储每个用户的隐私首选项。BlockSEC 存储用户的数据。最后,BlockTST 管理并验证有关消费者-生产者与消费者-公司之间的能源贸易的信息。为了保证区块链之间的通信,需要通过智能合约维护多链数据一致性。本文吸收了用智能合约自主决策的思想,基于改进的区块链结构和智能合约实现出租车调度系统。\par

\subsection{树状区块链的特性}
车载自组网具备低成本、低时延以及地理位置天然联系的特性,在道路安全等应用提供了重要支持\upcite{2020Application}。区块链作为一种新型的互联网模式,具备去中心化、不可篡改、可追溯等特点,与车联网的联系愈发紧密\upcite{2021Blockchain}。但实际应用上主要包括两个问题,首先,区块链技术通常要求可靠的网络接入,而车载自组网高动态性的接入方式引发的网络不稳定对区块链的应用造成了极大的挑战。其次,车载自组网中的信息通常与地理信息天然绑定,而传统区块链并不支持此特性。因此,研究将地理位置与区块链融合的树状区块链[周畅论文]就是为上述灵活性和地理区域问题提供一种可能的解决方案。\par

由于全局同步与共识速度的限制,传统区块链不能很好的适应车载自组网的高动态性,并且不能保证车辆与路侧节点的稳定连接,这将会极大影响到区块链共识机制的可用性。车载自组网与地理信息天然相联,并且车辆通常只需要关注特定区域的信息,而传统区块链记录的交易信息不包含位置,无法根据地理区域查询交易,且传统区块链全局同步的设计会造成网络资源的大量浪费,并且传统区块链无法根据地理信息高效地查询账户和交易相关的数据。树状区块链对区块链的结构进行修改与优化,使其更能适应车载自组网特性。树状区块链是一种基于位置的区块链[周畅论文],首先,根据 GeoHash 地理编码与地理区域层次关系建立树状结构的区块,同时研究树状结构地理信息相关数据查询速度的优化机制,从而提升区块链在车载自组网中的吞吐量以及与地理信息相关数据的检索速度。\par
GeoHash 是一个能表示任意精度的高效地理编码系统,它使用二分法将指定区域分为网格,每个网格由唯一的编码表示,网络的大小对应的层次不同,编码越长对应的网格越小,层次越低。树状区块链使用 GeoHash 地理编码作为基本数据编码方式。建立 GeoHash 与区块以及交易的索引,从而依据 GeoHash 编码直接获取区域交易内容,同时,GeoHash 编码可以通过前缀对区域信息进行绑定,因此通过 GeoHash
编码可以很自然地建立区域层次与子链的联系。因此本文选用树状区块链平台来实现分布式的出租车调度系统。

\section{GeoHash地理信息}
由于传统计算两个经纬度所表示坐标点距离时需要使用球面距离公式,若在以 GeoHash 为坐标表示的系统中沿用这套算法,则丧失了 GeoHash 带来的计算简便性。利用GeoHash编码的特点进行距离计算,避免了复杂的三角函数和球面计算,并且适用于对小数支持较弱、不提供复杂数学函数计算支持的区块链智能合约编写语言 Solidity。
简单参考董斌的论文。
\subsection{基于GeoHash的地图}
将矢量地图数据以GeoHash形式存到区块链上的原理、优点分析和改进空间。
\subsection{GeoHash几何计算原理}
介绍GeoHash数格子进行几何距离计算的原理。
\subsection{leaflet矢量地图渲染框架}
介绍leaflet矢量地图渲染框架,以及其对GeoHash数据的支持情况。

\section{出租车调度系统}
介绍出租车调度系统的研究现状,指出中心化系统的不足,容易引发信息垄断、信息泄露和恶性竞争。
介绍为解决这种问题提出的研究模型,以及优缺点分析。
分析几种平台上实现出租车调度系统的平台优缺点,以及自己采用平台的优点。
\section{导航算法}



介绍导航算法的发展历史和研究现状,分类简要介绍几种导航算法的优缺点,解释为什么有的导航算法不适合应用于静态网格道路,指出区块链应用缺乏导航算法的支持,以及在区块链中将导航算法与GeoHash结合的优点。



