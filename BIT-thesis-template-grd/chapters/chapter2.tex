%%==================================================
%% chapter02.tex for BIT Master Thesis
%%==================================================
\chapter{相关工作}
本章对本文研究内容的相关工作进行简要介绍。首先是基于以太坊的树状区块链;第二是GeoHash地理信息,第三是出租车调度系统的发展现状,第四是导航算法(简单描述)

其一是国内外现有路况计算算法,其二是分布式计算领域的计算框架和算法设计

2.1 路况计算
2.1.1 基于浮动车轨迹的路况计算算法
两篇基于瞬时速度的文献,瞬时速度不一定是可靠的,比如路口拥堵或红绿灯,而且采样率低。
低采样率:两篇文献采用平均速度,一篇采用历史道路速度加权,缺点在于不考虑路口,且轨迹点间可能包含多条道路。
一篇文献考虑了路口的影响,但是不能估计车辆的路口转向时间,且路况时效性不高。
关于路口转向,[27][28]。[11]将一部分道路路况分离到路口,张禹[17]根据历史路况得出道路速度和估计路口转向时间,然后将每辆车的真实时间加权分配。缺点是没有对立交桥进行区分处理。

本文在张禹[17]的基础上将立交桥纳入考虑范围。

2.1.2 路口的定位与范围划分
[15][26]将路口范围设置为统一值。[11]假设两个轨迹点间只能跨越一个路口,局限性大。[17]以三元组描述路口转向时间,解决两点之间跨越多个路口的情况。
[29][30]对立交桥进行几何学提取,缺点是需要依赖遥感和图形库。[31]通过聚类对复杂交叉口进行定位,但没有应用于城市整体。本文将立交桥抽象为路口,利用密度聚类对立交桥进行自动化定位。[31]将立交桥划分为圆形,不适合形状不规则的立交桥。本文使用道路连接点构成路口,将立交桥和路口一并存入数据库考虑。

2.2 分布式计算
2.2.1 分布式框架选择

2.2.2 分布式算法设计

2.3 本章小结


\section{基于以太坊的树状区块链}
介绍区块链的性能瓶颈,引出树状区块链的研究目的、简要介绍原理。
\subsection{传统区块链的特点和不足}
\subsection{树状区块链的特性}

\section{GeoHash地理信息}
由于传统计算两个经纬度所表示坐标点距离时需要使用球面距离公式,若在以 GeoHash 为坐标表示的系统中沿用这套算法,则丧失了 GeoHash 带来的计算简便性。利用GeoHash编码的特点进行距离计算,避免了复杂的三角函数和球面计算,并且适用于对小数支持较弱、不提供复杂数学函数计算支持的区块链智能合约编写语言 Solidity。
\subsection{基于GeoHash的地图}
将矢量地图数据以GeoHash形式存到区块链上的原理、优点分析和改进空间。
\subsection{GeoHash几何计算原理}
介绍GeoHash数格子进行几何距离计算的原理。
\subsection{leaflet矢量地图渲染框架}
介绍leaflet矢量地图渲染框架,以及其对GeoHash数据的支持情况。

\section{出租车调度系统}
介绍出租车调度系统的研究现状,指出中心化系统的不足,容易引发信息垄断、信息泄露和恶性竞争。
\section{导航算法}
介绍导航算法的发展历史和研究现状,分类简要介绍几种导航算法的优缺点,指出区块链应用缺乏导航算法的支持,以及在区块链中将导航算法与GeoHash结合的优点。
