%%==================================================
%% chapter01.tex for BIT Master Thesis
%% modified by yang yating
%% version: 0.1
%% last update: Dec 25th, 2016
%%==================================================
\chapter{基于GeoHash的导航算法}
为了完善去中心化的出租车调度系统,需要在智能合约端实现后台的导航算法。
\section{导航算法对比}
导航算法的提出和发展由来已久,有适合在未知地图环境下运行的启发式导航算法,可以应用在智能机器人和无人车等领域,此外,还有可以在已知地图信息的情况下,利用已有的矢量地图数据规划出最短路径的导航算法,可以应用在地理信息实时更新的交通系统中。
\subsection{导航算法的发展种类}
启发式导航算法,适合在不知道地理信息的情况下进行主观的路径探索,这种算法应用在机器人的自动寻路、游戏中的AI角色寻路等场景;路径导航算法,在已知地理信息的情况下进行最短路径的规划,可以应用在车载应用的导航、公共交通实时导航等环境中。
\subsection{矢量地图路径导航算法的性能对比}
理论分析astar算法相比djkstra路径导航算法的性能优劣,对GeoHash的适配性,阐述选择astar路径导航算法作为原型的原因。

……\upcite{Takahashi1996Structure,Xia2002Analys is,Jiang1989,Mao2000Motion,Feng1998}

\section{基于GeoHash的导航算法设计}
\subsection{astar导航算法的原理}
详细解释astar导航算法的原理。
\subsection{基于GeoHash的astar导航算法设计}
基于astar导航算法设计出支持GeoHash格式的导航算法的原理。