%%==================================================
%% thanks.tex for BIT Master Thesis
%% modified by yang yating
%% version: 0.1
%% last update: Dec 25th, 2016
%%==================================================

\begin{thanks}

首先,衷心感谢我的导师陆慧梅老师和向勇老师。两位老师在我因迷茫而失落的时候,像灯塔一样为我指明了方向。在我遇到困难而不知所措的时候,教会了我调节心态和解决问题的有效方法。导师对我论文的研究方向提供了宝贵的意见和细心的指导,教会了我如何自我发现和解决问题,教会了我严谨的批判性思维,培养了我在科研工作中的逻辑思维能力和梳理总结能力。导师耐心的指导和教诲是我读研期间最珍贵的收获。

感谢同门的周畅师姐和万琦玲师妹。她们在我论文撰写过程中给予了很多帮助,提出了许多有益的改善性意见。我在科研方向选择和实验的具体细节上,都获得了周畅师姐的诸多帮助。我还要感谢1036实验室的所有老师同学。感谢王馨茹师姐在我一次次情绪失落时对我的安慰和鼓励。感谢雷鑫晨同学带我去健身房强身健体,拓展经历。感谢于长鑫同学教会我容器技术的使用,提高了我做实验搭环境的效率。感谢李祥潮同学和刘金田同学分享的求职经验,让我感受到了大佬们的眼界和本领。感谢曾秋阳同学和孟玲同学的解忧陪聊和关心。实验室的大家在学习过程中互相学习,互相帮忙,共同度过了一段美好难忘的时光。此外,我要感谢辅导员王柳婷对我的帮助,不论是学习课程还是完成毕设期间,她都为学生事务尽职尽责,为我们每一位学生排解烦恼和解决问题。

感谢我的朋友刘浩磊、丁波文和任浩,我们共同拥有一起考研、一起探讨人生目标、一起相互鼓励这些深刻的人生经历。我还要感谢我的舍友许泽松同学,在课程作业中他是得力的伙伴,在一起求职时他是消息灵通的战友。感谢邱柯铭同学,在我遇到困难的时候给了我很多支持和帮助。同时,我也要感谢我的工作参考文献中的作者们。他们的深入研究鞭辟入里,让我在研究课题中找到了很好的出发点。

最后,谢谢论文评阅老师们的辛苦工作,感谢陆慧梅老师、杨松老师、段春晖老师对我学习上的关心和帮助。衷心感谢一直以来陪伴着我的家人、朋友和同学们。正是他们在背后坚定的支持和帮助,我才得以顺利完成此论文。两年时光如白驹过隙,前路漫漫,吾将上下而求索。

\end{thanks}
