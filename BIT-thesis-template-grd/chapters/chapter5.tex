%%==================================================
%% chapter05.tex for BIT Master Thesis
%%==================================================
\chapter{参数设置和系统测试}
出租车调度系统的实验环境、本章的工作整体介绍。
% 张禹:
% 实验环境
% 调节OHMMM算法边界值(一个流程图,三个数据图)
% 近似算法参数优化,设置容量上限为200(一图一表)
% 直接路况精度
%   差错率:带转向延迟的差错率更低,且不带来计算开销(两图)
%   分析了三种车辆异常行为
% 间接补充的路况精度
%   补充路况能够有效提高路况覆盖率(两图)
% 路况结果分析
%   路况统计比轨迹点统计更平滑(一图)
%   统计道路速度(一图)
%   统计路口转向延迟,和道路速度对比关系,证明正确性(一图)
%   展示矢量地图路况(一图)
% 总结

\section{数据分析工作}

1. 出租车GPS数据字段包括是否载客的状态,利用对出租车是否载客的分析可以得到出租车的分布特点和乘客的分布特点。

2. 分析真实的出租车数据,讨论不同时段的车辆和乘客在城区的分布情况(既考虑了时间因素也考虑了位置因素),依此设置空车和乘客的初始化位置和比例,为下一步系统仿真测试做铺垫。

\section{参数设置实验}

\subsection{astar导航算法参数实验}
1. 用前缀匹配算法优化GeoHash的距离计算,然后用GeoHash距离计算得到的真实地理距离作为astar路径规划算法的启发函数结果。
2. 用GeoHash解码数格子,用东西和南北格子差相加(曼哈顿距离)作为启发函数结果,调节耗散函数的参数来适配启发函数的结果。
(3. 用中间状态点和终点GeoHash位置相同前缀的数量作为启发函数的结果,误差太大。)
4. 选取不同的起止点,当起止点的直线距离在2km、5km、10km、15km、20km时,两种算法下路径规划结果的实际路径长度(平均值)和算法消耗时间(平均值)、或者区块链的gas消耗量做对比,确定更准确的算法和参数。





\subsection{对树状区块链的区域调度等区块链特征进行实验}

利用真实分布数据,模拟车辆和乘客在城区的位置分布
1. 对比五位区域查询到的车辆和六位区域查询到的车辆距离乘客上车位置的平均距离,分析哪种区域查询方法更适合车辆调度。
2. 对比全局查询和区域查询的系统响应时间,说明区域查询会提高系统效率。
3.对比全局查询和区域查询的区块链gas消耗,说明区域查询会提高系统效率。
4.一个区块链节点管理所有区域,和多个区块链节点分别管理区域,对比两种情况下后台的响应时间、对比节点存储的数据量、服务乘客和车辆的数目(负载情况)。

\section{系统测试实验}

展示近距离导航和远距离导航的路径

\subsection{真实地图数据测试}
真实地图的导航结果测试。

收集数据:
1. 在不同环境下(请求频繁或者稀疏)乘客平均等待时间、车辆平均空车时间(即处于未载客状态的时间),或者百分率,验证区域调度相比全局调度的优越性,系统将区块链与地理信息结合是有意义的。
2. 在不同环境下(请求频繁或者稀疏)系统对乘客请求的响应时间,响应时间在一定阈值范围内,可验证系统的合理性和正确性。