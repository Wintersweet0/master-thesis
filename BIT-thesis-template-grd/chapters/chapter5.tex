%%==================================================
%% chapter05.tex for BIT Master Thesis
%%==================================================
\chapter{参数设置和系统测试}
出租车调度系统的实验环境、本章的工作整体介绍。
% 张禹:
% 实验环境
% 调节OHMMM算法边界值(一个流程图,三个数据图)
% 近似算法参数优化,设置容量上限为200(一图一表)
% 直接路况精度
%   差错率:带转向延迟的差错率更低,且不带来计算开销(两图)
%   分析了三种车辆异常行为
% 间接补充的路况精度
%   补充路况能够有效提高路况覆盖率(两图)
% 路况结果分析
%   路况统计比轨迹点统计更平滑(一图)
%   统计道路速度(一图)
%   统计路口转向延迟,和道路速度对比关系,证明正确性(一图)
%   展示矢量地图路况(一图)
% 总结

\section{参数设置实验}
\subsection{astar路径规划算法参数实验}
1. 用前缀匹配算法优化GeoHash的距离计算,然后用GeoHash距离计算得到的真实地理距离作为astar路径规划算法的启发函数结果。
2. 用GeoHash解码数格子,用东西和南北格子差相加(曼哈顿距离)作为启发函数结果,调节耗散函数的参数来适配启发函数的结果。
(3. 用中间状态点和终点GeoHash位置相同前缀的数量作为启发函数的结果,误差太大。)
4. 选取不同的起止点,当起止点的直线距离在2km、5km、10km、15km、20km时,两种算法下路径规划结果的实际路径长度(平均值)和算法消耗时间(平均值)、或者区块链的gas消耗量做对比,确定更准确的算法和参数。

距离计算优化实验

这部分工作是对本文提出的距离计算优化的效果进行实验验证。考虑出租车调度系统的规模,探究了20km内的起点和终点GeoHash进行距离计算的运算效果。收集优化前后算法结果的响应时间和区块链gas消耗作为对比性能的参考,结果如图:


由图分析可知,在20km内不同距离的起止点规模下,前缀优化算法在响应时间上对距离计算的性能提升为28$\%$左右,算法优化后的gas消耗也有明显减少,值得注意的是,算法的计算速度与两点之间距离的远近没有明显的相关性。考虑到GeoHash块的大小,4位GeoHash块的南北长约19.54千米,东西长约28.96千米,因此距离在20km的范围内的两个起止点GeoHash通常前3位及以上的字符是相同的,因此进行计算优化后的优化效果比较明显。但考虑到边界情况,20km范围内的两点GeoHash也有前三位的字符不同的时候。算法优化前后两点GeoHash前缀字符相同的位数与算法运行性能的关系如下图(响应时间对比、gas消耗两图):


由图中结果可知,在考虑两点GeoHash的特殊情况下,两点之间的GeoHash相同前缀字符的位数会影响距离计算的速度,相同的前缀越长,优化计算的效果越明显。



算法正确性实验
  路径规划算法的路径规划结果正确性展示图(一图,或一模拟图一真实图)



路径规划算法的参数实验
  路径规划算法遍历的路段数目、算法的gas消耗、算法的运行耗时(三张图)

  本部分针对不同计算精度下的路径规划算法准确度进行了实验,旨在保证路径规划算法的正确性和结果精度的情况下,提高路径规划算法的计算性能。manhattan函数的距离计算精度可以通过调节GeoHash的精度来改变,路径规划算法进行计算的精度最高到达10位GeoHash。图中nopre代表没有经过距离计算速度优化且计算精度最高(即基于10位GeoHash的格子大小进行计算)的路径规划算法,pre_10代表距离计算速度优化后且计算精度最高(基于10位GeoHash)的路径规划算法,pre_9代表基于9位GeoHash格子大小的精度进行计算的路径规划算法,以此类推,pre_7代表基于7位GeoHash格子大小的精度进行计算的路径规划算法。

  由实验结果可以看出,有无距离计算速度上的优化,对路径规划算法遍历的路段数目影响不明显。同时可以发现,路径规划算法基于的距离计算精度,对路径规划算法遍历路段的数目有一定的影响。值得注意的是,计算距离时基于的GeoHash格子规模扩大后,算法遍历的路段数量略有下降。这是因为格子规模扩大后,启发函数对各路段相距终点的距离计算结果区分度降低,与终点距离相同的中间路段会增多,而算法在遍历到与终点距离相同的各个中间路段时,不会将其重复加入到算法队列中,如此便降低了算法运行时遍历的路段数目。



  区块链执行路径规划算法的gas消耗,随着起止点距离的增加会不断增大,主要是因为起止点之间的路段数目增多,算法遍历的路段数增加,导致后台运算的开销增大。实验收集的数据表明,在没有距离计算优化时,GeoHash的路径规划算法消耗的gas数目明显高于优化后的算法,并且随着起止点之间距离的增加,gas消耗的增长趋势也显著增大。

  ​在GeoHash距离计算的格子规模不同时,可以观察到,随着GeoHash精度位数的降低,同一对起止点进行路径规划算法的gas消耗会下降,这是因为,距离计算基于格子的规模变大,也就是GeoHash位数减小后,后台的运算量会随之降低,但同时也牺牲了距离计算的精度,用距离计算的精度换取了运算效率的提升。但也不能一味扩大格子的规模,因为7位GeoHash的范围已经达到了50—100米。可以覆盖一个小型路口,如果继续降低GeoHash的位数来增大规模,会明显影响路口处的路径规划精度。(实验验证)

  监测路径规划算法的运行耗时,以发出起止点请求到收到路径规划结果为止。

  随着起止点距离的增加,路径规划功能的响应时间会不断增大。在30km的起止点规模下,每一次启发函数的距离计算时间一般不会根据两点之间距离变化受到明显影响。起止点距离增加后,之所以路径规划的响应时间也会增加,主要是因为连接起止点之间的路段的个数变多,导致算法需要遍历的路段个数变多,返回路径规划结果的时间也会随之增加。

  根据数据结果可以看出,在距离计算优化前,路径规划算法的响应时间是明显高于距离计算优化后的路径规划响应时间的,距离计算优化后,启发函数计算每条路段与目的地的距离时,运算速度会增加。此外,在GeoHash距离计算的精度不同时,随着GeoHash精度位数的降低,同一对起止点进行路径规划算法的响应时间会降低。

  这一方面是因为,距离计算基于的GeoHash精度位数降低,即GeoHash格子规模变大后,进行距离计算的运算量会随之降低,从而缩短了路径规划返回结果的时间。

  另一方面,启发函数计算出各路段相距目的地距离的区分度降低,与终点距离相同的中间路段不会重复加入到算法队列中,提高了算法的效率,从而降低了响应时间。

  综合考虑运算效率和结果准确度,选取距离计算优化后,GeoHash精度为8时(即prefix_8)的路径规划算法,可在满足路径规划结果准确度的条件下达到最高的运算效率。距离计算优化提升了()的计算效率,调节计算精度提升了()的计算效率。



\subsection{对树状区块链的区域调度等区块链特征进行实验}

利用真实分布数据,模拟车辆和乘客在城区的位置分布
1. 对比六位区域查询到的车辆和六位邻居区域查询到的车辆距离乘客上车位置的平均距离,分析哪种区域查询方法更适合车辆调度。
2. 对比全局查询和区域查询的系统响应时间,说明区域查询会提高系统效率。
3.对比全局查询和区域查询的区块链gas消耗,说明区域查询会提高系统效率。
4.一个区块链节点管理所有区域,和多个区块链节点分别管理区域,对比两种情况下后台的响应时间、对比节点存储的数据量、服务乘客和车辆的数目(负载情况)。


区域调度算法的实验
  对比6位区域调度和6位邻居区域调度的车乘匹配效果,响应时间对比、车辆到乘客的距离对比(两图)
  

  高峰和低峰环境下的对比结果(一图)

  可以观察到,所有乘客进行6位区域调度的响应时间均明显低于全局调度的响应时间;

  不同位置乘客,其进行全局调度请求的响应时间区别较大。这是因为,查找距离乘客最近的空车时,全局调度需要遍历的车辆较多,若后台未能先找到距离乘客较近的车辆,那么后续找到更近距离的空车后,会多次更改智能合约内部的数据状态,导致响应时间延长。故不同位置的乘客进行全局调度请求时,系统的响应时间区别较大。

  相比全局调度,相同乘客进行6位区域调度的响应时间低于全局调度的比例为100$\%$,这说明了在出租车调度系统中应用地理位置区块链的正确性和有效性。

  可以观察到,每个乘客在两种环境下的调度请求结果,有时是全局调度的车辆更远,有时是区域调度的车辆更远,但二者绝大部分的距离结果相差并不大(平均起来,每位乘客与其匹配到的车辆的距离,在6位GeoHash的区域调度时,只比全局调度远$\%$。

  但值得注意的是,全局调度中会有乘客匹配到明显较远的车辆,其距离是乘客平均等车距离的7.63倍、6.23倍,让少数乘客付出了更多的等待时间。这说明,相比全局调度而言,区域调度能保证匹配到的出租车不会距离乘客太远,从而提高了所有乘客的打车体验,同时还能保持平均调度距离不明显增加。

  但区域调度显著降低了系统的响应时间,同时保证了所有乘客的打车体验,证明了地理位置区块链的应用相比传统区块链具有明显优势。

  但基于6位GeoHash的区域调度也有易见的缺点:将可选车辆的范围限制在了6位GeoHash区域的范围内,如果乘客的位置在6位GeoHash区域的边缘,则在该6位区域外的相邻区域可能有比原区域内距离乘客更近的空车。

  六位GeoHash邻居区域调度:

  以乘客位置所在的6位GeoHash区域为中心,根据算法获得其周围的8个邻居区域,地理位置区块链根据这九个区域查询到区域内对应的车辆账户,然后从这些车辆中筛选出距离乘客最近的空车。此调度方法克服单独的6位GeoHash区域调度的边界现象,让车乘匹配的结果更合理。

  单独的6位区域调度只能调度到自己所属地理区域内的车辆,当乘客所处的位置位于6位GeoHash地理区域的边缘时,如果这个区域内车辆的数目不够,有部分乘客的打车请求不会被满足,但实际上乘客附近的其它6位GeoHash区域中有空车,而且相距乘客的距离也合适,但因为地理划分的问题没有被考虑。6位邻居区域调度可以根据乘客的位置调度到乘客所处区域和周围邻居区域内最近的空车,减少GeoHash划分的地理区域的限制,能够更好地满足乘客的打车需求。

实验数据表明,6位邻居调度调度结果的车乘平均距离,比单独6位GeoHash区域调度结果的车乘平均距离更近,这是因为邻居区域调度优化了单独区域调度的边界限制,乘客在6位GeoHash地理区域边缘可以匹配到邻居区域内实际距离更近的车辆,而不是匹配到本区域内更远的车辆。



系统测试实验

知行合一知行合一

系统测试实验从实际应用的方面探究本文所提出方法的实用性和正确性。分为模拟道路的系统实验和真实道路的系统实验两部分。

【贾兴无】分析了工作日居民选择出租车出行的供需图,其工作统计出,在高峰时段,乘客的出行需求数目可达同时段空驶出租车供给数目的约1.5倍,在平峰时段,乘客的出行需求数目仅仅略高于空驶出租车的供给数目。贾的工作将城市核心区根据不同的土地性质划分为居住区、商业区等交通小区,交通小区的直径在0.5km到2.5km之间,工作日各交通小区的高峰某时刻的出租车需求频数在20至200之间,高峰期每小时的出租车需求频数在60至600之间,本文的系统测试
选取2.5km规模的区域进行调度系统的初始化和运行,观察本文所提出的理论工具的实际效果。

  系统运行时的打车响应时间对比(一图)
  系统运行时的车乘距离对比(一图)
  传统和树状区块链下,适应的规模内,车辆载客时间的占比对比(一图)

  树状区块链下,不同场景的系统性能表现(比如不同乘客请求规模下的接单率、响应时间等)、两到三图






% \section{数据分析工作}

% 1. 出租车GPS数据字段包括是否载客的状态,利用对出租车是否载客的分析可以得到出租车的分布特点和乘客的分布特点。

% 2. 分析真实的出租车数据,讨论不同时段的车辆和乘客在城区的分布情况(既考虑了时间因素也考虑了位置因素),依此设置空车和乘客的初始化位置和比例,为下一步系统仿真测试做铺垫。

\section{系统测试实验}

展示近距离路径规划和远距离路径规划的路径

\subsection{真实地图数据测试}
真实地图的路径规划结果测试。

收集数据:
1. 在不同环境下(请求频繁或者稀疏)乘客平均等待时间、车辆平均空车时间(即处于未载客状态的时间),或者百分率,验证区域调度相比全局调度的优越性,系统将区块链与地理信息结合是有意义的。
2. 在不同环境下(请求频繁或者稀疏)系统对乘客请求的响应时间,响应时间在一定阈值范围内,可验证系统的合理性和正确性。