%%==================================================
%% chapter05.tex for BIT Master Thesis
%%==================================================
\chapter{参数设置和系统测试}
出租车调度系统的实验环境、本章的工作整体介绍。
张禹:
实验环境
调节OHMMM算法边界值(一个流程图,三个数据图)
近似算法参数优化,设置容量上限为200(一图一表)
直接路况精度
  差错率:带转向延迟的差错率更低,且不带来计算开销(两图)
  分析了三种车辆异常行为
间接补充的路况精度
  补充路况能够有效提高路况覆盖率(两图)
路况结果分析
  路况统计比轨迹点统计更平滑(一图)
  统计道路速度(一图)
  统计路口转向延迟,和道路速度对比关系,证明正确性(一图)
  展示矢量地图路况(一图)
总结



\section{参数设置实验}
astar导航算法和区域调度算法的参数调节
\subsection{astar导航算法参数}
\subsection{区域调度算法参数}

\section{系统测试实验}

\subsection{模拟双行道正确性测试}
模拟双行道的导航结果测试。
\subsection{真实地图数据测试}
真实地图的导航结果测试。