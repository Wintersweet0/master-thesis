%%==================================================
%% chapter01.tex for BIT Master Thesis
%%==================================================
\chapter{绪论}
\section{本论文研究的目的和意义}

随着城市交通的逐渐发展,道路网络的复杂度以及车辆保有量日益增长,交通基础设施的建设无法满足需求,给交通流量的管理带来困难。在智能交通系统中,自动化的出租车调度系统可以有效地满足实时的出行需求,为智能交通系统的完善提供技术支持,有效地满足城市交通需求以及交通流量的管理与诱导,能够有效提高用户出行效率。(研究出租车调度系统的目的和意义)在车辆调度领域,目前已有集中式的自动化出租车调度系统\upcite{魏玉光2021自动驾驶出租车调度方法及调度系统}。\par
信息的去中心化管理是一项关键技术。设计去中心化的出租车调度系统可以保障用户信息不被滥用,避免恶性竞争,在提高交通系统中乘客乘车效率的同时,也兼顾了出租车司机收入的公平。例如已有为缓解机场压力而设计的机场出租车调度模型,从机场管理部门的角度统筹部署机场出租车调度管理系统,在一定程度上避免了私营企业通过集中式后台管理的软件产品进行不正当牟利,提高机场和市政部门的工作效率\upcite{邢智璇2021兼顾效率与公平的机场出租车调度管理系统研究}。\par
区块链可以用作去中心化的出租车调度系统的开发平台\upcite{2020Blockchain},区块链是一种去中心化的共享账本,它可以安全地将简单,连续和经过验证的数据存储在系统软件中。与传统技术相比,区块链具有以下三个优势:\par
一是它不能被伪造,所以更安全。每个人都必须验证自己的身份,然后才能根据专用安全通道访问数据库,添加或获取数据,并保留历史访问数据。\par
第二是具有很强的实用性和可信性。每个节点都将维护一个详细数据账本,并且该数据组仍处于不同对象的控制之下,并且根据共识算法将数据维持在高度一致的水平,增强了系统的健壮性。\par
第三,它具有智能合约和全自动执行功能。智能合约具有完全透明,可靠,强制执行和全自动执行的优势,可以支持构建去中心化的后台系统。\par
利用区块链这一去中心化后台开发自动化的出租车调度系统,可以同时拥有信息保密、系统安全健壮、利益分配公平公正这些优点,有助于解决人们对高可信性的出行软件的需求问题,为推动智慧交通系统的建设添砖加瓦。


\section{国内外研究现状及发展趋势}
简单介绍自组网应用中的导航应用研究现状,以及局限性分析
\section{论文的研究内容、贡献和组织结构}
引出本文研究内容,讲出工作贡献和论文的组织结构
\subsection{论文的研究内容}
(1)为优化区块链对地图数据的处理性能,降低计算量,引入GeoHash矢量地理信息存储,并为前端矢量地图数据渲染工具——leaflet添加支持GeoHash格式的放缩和拖动功能。\par
(2)为解决中心化的打车软件对用户信息进行违规利用、利用信息差进行恶性竞争等问题,本文基于树状区块链设计并实现了去中心化的出租车调度系统。\par
(3)实现出租车调度系统的核心是实现导航算法和车乘匹配算法,然而,树状区块链缺乏基于矢量地理数据的导航算法支持,为解决这个问题,本文在基于以太坊的树状区块链平台设计并实现了基于GeoHash矢量地理数据的导航算法,并在此基础上实现了基于树状区块链的区域调度车乘匹配算法。\par
(4)针对基于GeoHash的距离计算方法,本文通过前缀匹配对算法逻辑进行了优化;同时,本文在对已广泛应用的导航算法进行了修改,使其支持GeoHash的距离计算逻辑,在导航算法中加入可调节的参数增强其适配性;对导航算法和区域调度车乘匹配算法的关键参数进行调优。
\subsection{论文贡献}
(1)完善leaflet工具对GeoHash格式矢量地图展示的支持。\par
(2)提升GeoHash距离计算算法的速度。\par
(3)设计出基于GeoHash的导航算法并进行参数调优。\par
(4)设计出基于树状区块链的区域调度车乘匹配算法并进行参数调优。
\subsection{论文的组织结构}
第一章,介绍导航应用研究现状以及去中心化调度系统对于智能交通和反垄断、反恶性竞争的意义。\par
第二章,对本文涉及的相关工作进行综述,包括基于以太坊的树状区块链、GeoHash地理信息、leaflet渲染工具、出租车调度系统、导航算法。\par
第三章描述系统的框架,包括服务端智能合约的设计结构,和浏览器客户端车辆和乘客的设计结构,以及系统的运行流程。\par
第四章详细介绍系统用到的各种技术,包括基于GeoHash的矢量地图存储和展示、基于GeoHash的导航算法设计、基于树状区块链的区域调度车乘匹配算法。\par
第五章,介绍了出租车调度系统的参数调优以及系统在真实数据下的工作状态。