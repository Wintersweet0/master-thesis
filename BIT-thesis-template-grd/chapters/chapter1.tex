%%==================================================
%% chapter01.tex for BIT Master Thesis
%% modified by yang yating
%% version: 0.1
%% last update: Dec 25th, 2016
%%==================================================
\chapter{绪论}
\section{本论文研究的目的和意义}

随着城市交通的逐渐发展,道路网络的复杂度以及车辆保有量日益增长,交通基础设施的建设无法满足需求,给交通流量的管理带来困难。在智能交通系统中,信息的去中心化管理是一项关键技术。其中去中心化的出租车调度系统可以有效地满足实时的出行需求,为智能交通系统的完善提供技术支持,有效地满足城市交通需求以及交通流量的管理与诱导,能够有效提高用户出行效率。(研究出租车调度系统的目的和意义)
……\upcite{Takahashi1996Structure,Xia2002Analys is,Jiang1989,Mao2000Motion,Feng1998}

\section{国内外研究现状及发展趋势}
简单介绍自组网应用中的导航应用研究现状,以及局限性分析
\section{论文的研究内容、贡献和组织结构}
引出本文研究内容,讲出工作贡献和论文的组织结构
\subsection{论文的研究内容}
(1)为优化区块链对地图数据的处理性能,降低计算量,引入GeoHash矢量地理信息存储,并为前端矢量地图数据渲染工具——leaflet添加支持GeoHash格式的放缩和拖动功能。\par
(2)为解决中心化的打车软件对用户信息进行违规利用、利用信息差进行恶性竞争等问题,本文基于树状区块链设计并实现了去中心化的出租车调度系统。\par
(3)实现出租车调度系统的核心是实现导航算法和车乘匹配算法,然而,树状区块链缺乏基于矢量地理数据的导航算法支持,为解决这个问题,本文在基于以太坊的树状区块链平台设计并实现了基于GeoHash矢量地理数据的导航算法,并在此基础上实现了基于树状区块链的区域调度车乘匹配算法。\par
(4)针对基于GeoHash的距离计算方法,本文通过前缀匹配对算法逻辑进行了优化;同时,本文在对已广泛应用的导航算法进行了修改,使其支持GeoHash的距离计算逻辑,在导航算法中加入可调节的参数增强其适配性;对导航算法和区域调度车乘匹配算法的关键参数进行调优。
\subsection{论文贡献}
(1)完善leaflet工具对GeoHash格式矢量地图展示的支持。\par
(2)提升GeoHash距离计算算法的速度。\par
(3)设计出基于GeoHash的导航算法并进行参数调优。\par
(4)设计出基于树状区块链的区域调度车乘匹配算法并进行参数调优。
\subsection{论文的组织结构}
第一章,介绍导航应用研究现状以及去中心化调度系统对于智能交通和反垄断、反恶性竞争的意义。
第二章,对本文涉及的相关工作进行综述,包括