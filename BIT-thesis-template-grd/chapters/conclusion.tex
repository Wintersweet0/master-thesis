%%==================================================
%% conclusion.tex for BIT Master Thesis
%%==================================================


\begin{conclusion}

智能汽车的广泛使用,使得车辆之间的联系越来越紧密。利用车联网,将智能车辆组织起来,分享和传递数据。车联网需要建立去中心化的网络,而区块链具备去中心化、不可篡改、共识机制等特点,所以将区块链技术用于车联网中是一种可行方案。然而,车联网需要与地理信息密切结合,且其中的节点具备移动性的特点。传统区块链的存储结构并不能够很好的解决上述问题。另外,车联网的大量应用都与地理位置信息和身份信息密切相关,而数据来源于开放的互联网平台,存在被篡改的风险,且传统的地理信息以经纬度的形式在矢量地图表示,不能很好地支持区域信息绑定和查询。现有的基于区块链的车辆调度研究工作,大多是针对基于区块链的身份验证机制和零知识信任机制进行研究,或是针对区块链节点在性能方面的处理能力进行研究,缺乏对地图数据、调度机制和业务场景进行实际应用的探索。

针对上述问题,本文完成了对区块链相关工作的调研,并对地理位置区块链的研究进行了介绍。同时,本文对地图信息的存储、展现和使用进行了相关研究,拓展了GeoHash矢量地图的相关功能。本文在区块链平台实现了继续GeoHash矢量地图的路径规划算法,基于地理位置区块链平台设计了出租车调度的匹配机制,并在此基础上实现了完整的出租车调度系统设计,然后将该系统与传统区块链环境下的调度系统作对比。接下来本文还对基于地理位置区块链的调度系统性能做了实验探究,以及在模拟环境和真实道路的环境下对系统的实用性进行了测试和验证。本文的主要工作总结如下:

1. 增加了GeoHashTile地图的功能,实现了Leaflet地图工具从屏幕坐标点到地理位置GeoHash点的投影过程,并在此基础上实现了对GeoHash矢量地图的放缩和拖动功能,优化了GeoHash矢量地图的展示效果,更有利于GeoHash矢量地图的研究和实际应用。

2. 优化了基于GeoHash的距离计算方法,优化了服务端的计算速度,降低了距离计算的gas开销,从而降低了区块链平台的计算压力,更有利于在区块链平台上进一步实现出租车调度系统。

3. 设计并实现了基于GeoHash矢量地图的路径规划算法,并根据GeoHash位置数据的特点对算法的计算参数进行优化,在保证路径规划结果正确性的同时,提高了算法的计算效率,降低了区块链平台的运算量,更有利于系统实现。

4. 设计并实现了基于地理位置区块链的区域调度方法,实验结果表明地理位置区块链环境下的区域调度方法比传统区块链环境下的全局调度方法具有更强的拓展性。地理位置区块链服务端支持的调度系统车乘交互规模是传统区块链的3倍左右。在其他条件相同的情况下,地理位置区块链环境支持的系统运行结果,其车辆的有效运行时间、乘客打车的运行时间、匹配到的车辆相距乘客的距离均有优化。本文对调度系统进行规模化测试的运行结果也均处于合理范围,证明了将GeoHash编码、地理信息和区块链相结合的思想,能够良好的适用于车联网的实际应用。

本文的工作仍有可拓展的研究方向和不足之处:
可以探究区块链的准入机制和零知识证明机制,在用户安全的层面加强系统的安全性。另外,为了加强出租车调度系统与实际地理信息的结合,可以将实时路况的维护与路径规划算法相结合,通过区块链本身的特点增强路况结果的安全性和可信性,同时让导航结果更具备实用性和参考意义。此外,可以在系统中加入对出租车空驶巡航的路线推荐功能,在解决各个时期的乘车资源分配问题方面做出探究,更有效地利用地理信息与区块链相结合的特点,提升交通系统的运行效率。

\end{conclusion}