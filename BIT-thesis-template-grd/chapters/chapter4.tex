%%==================================================
%% chapter04.tex for BIT Master Thesis
%%==================================================
\chapter{系统的工具和算法原理}
本章详细介绍系统的工具、功能、算法开发工作和原理。

\section{基于GeoHash的矢量地图展示}
对系统开发过程中将基于GeoHash的矢量地图信息存储在区块链和展示在浏览器端的开发工作和原理做介绍。
\subsection{基于GeoHash的地图在区块链上的存储}
将矢量地图数据以GeoHash形式存到区块链上的原理、优点分析和改进空间。
\subsection{基于GeoHash的矢量地图实现放缩和拖动功能}
描述放缩和拖动功能的实现原理,解释GeoHash矢量地图的渲染步骤。

\section{基于GeoHash的几何计算优化}


由于传统计算两个经纬度所表示坐标点距离时需要使用球面距离公式,若在以 GeoHash 为坐标表示的系统中沿用这套算法,则丧失了 GeoHash 带来的计算简便性。利用GeoHash编码的特点进行距离计算,避免了复杂的三角函数和球面计算,并且适用于对小数支持较弱、不提供复杂数学函数计算支持的区块链智能合约编写语言 Solidity。
\subsection{GeoHash几何计算原理}
具体介绍GeoHash数格子进行几何距离计算的原理。
\subsection{GeoHash几何计算方法的优化}
介绍前缀匹配的思想对GeoHash计算速度的优化原理。

\section{基于GeoHash的导航算法}
为了完善去中心化的出租车调度系统,需要在智能合约端实现后台的导航算法。
\subsection{导航算法的发展种类}
导航算法的提出和发展由来已久,有适合在未知地图环境下运行的启发式导航算法,可以应用在智能机器人和无人车等领域,此外,还有可以在已知地图信息的情况下,利用已有的矢量地图数据规划出最短路径的导航算法,可以应用在地理信息实时更新的交通系统中。启发式导航算法,适合在不知道地理信息的情况下进行主观的路径探索,这种算法应用在机器人的自动寻路、游戏中的AI角色寻路等场景;路径导航算法,在已知地理信息的情况下进行最短路径的规划,可以应用在车载应用的导航、公共交通实时导航等环境中。
\subsection{矢量地图路径导航算法的性能对比}
理论分析astar算法相比djkstra路径导航算法的性能优劣,对GeoHash的适配性,阐述选择astar路径导航算法作为原型的原因。
\subsection{astar导航算法的原理}
详细解释astar导航算法的原理。
\subsection{基于GeoHash的astar导航算法设计}
详细解释astar导航算法的原理。介绍基于astar导航算法设计出支持GeoHash格式的导航算法的原理。

\section{基于树状区块链的区域调度车乘匹配算法}
\subsection{树状区块链对区域信息的查询}
介绍树状区块链对区域信息的查询原理。
\subsection{区域调度车乘匹配算法}
介绍在树状区块链对区域信息查询的基础上实现的区域调度车乘匹配算法原理、对并发请求的冲突解决逻辑。

\upcite{2016A}考虑正常交通对出租车的影响,提出了一种出租车调度系统,该系统使用实时交通状况将空置的出租车与最近的(在旅行时间方面)等候的乘客相匹配。