%%==================================================
%% abstract.tex for BIT Master Thesis
%% modified by yang yating
%% version: 0.1
%% last update: Dec 25th, 2016
%%==================================================

\begin{abstract}

车载自组网是在交通环境参与者间构建的开放式网络,可以为用户提供去中心化的数据传输能力。基于车载自组网,可以实现事故预警、辅助驾驶、道路交通信息查询、车间通信和网络接入服务等应用。研发这些应用需要地理信息和交通数据的支持,但信息的垄断会引发不正当牟利和恶性竞争。针对这一问题,本文利用部署在车载自组网上的区块链网络,基于GeoHash矢量地图和基于以太坊的树状区块链平台,开发了一套出租车调度和导航系统以完成出租车的去中心化调度。首先,本文选用GeoHash作为系统中统一的位置信息表示方法,在系统实现上采用浏览器与智能合约相结合的方式,在智能合约端开发了基于GeoHash的路径导航算法和车辆的区域调度算法,解决了车乘分配时的并发冲突问题。在浏览器端实现车辆和乘客的数据采集和乘车业务完整流程的设计。系统充分利用了区块链的性质保证车辆信誉数据的安全性、可溯性和在网络内的同步性。利用GeoHash在地理信息上的计算特性对算法速度进行了优化,并进行了优化后的实验验证工作。最后调节系统的关键参数进行性能优化并进行实验验证,通过真实的地图数据验证了此出租车调度系统的可行性。

\keywords{区块链; 导航; GeoHash }
\end{abstract}

\begin{englishabstract}

VANET is an Ad­hoc networks between participants in trafic and providing decentralized data transmission service. VANET can be uesd in application like accident
warning, drive assist system, trafic information service and Inter­Vehicle Communication. The development of these applications requires the support of geographic information and traffic data, but the monopoly of information will lead to unfair profit-making and vicious competition. In response to this problem, this paper uses the blockchain network deployed on the in-vehicle ad hoc network, based on the GeoHash vector map and the Ethereum platform, to develop a taxi dispatch and navigation system to complete the decentralized dispatch of taxis. First, this thesis use GeoHash for storage and calculation of position. The system in this thesis is made up of browser­side programs and smart contract. On the smart contract side, a GeoHash-based route navigation algorithm and a vehicle regional managing algorithm were developed, which solved the problem of concurrency conflicts in the allocation of vehicles and passengers. The browser side realize vehicle and passenger data collection and design the complete process of ride-hailing business. Blockchain makes this system safe, traceable and synchronized. The speed of the algorithm is optimized by using GeoHash's computing characteristics on geographic information, and the optimized experimental verification work is carried out. Finally, the key parameters of the system are adjusted for performance optimization and experimental verification. The feasibility of the taxi dispatching system is verified by real map data. 

\englishkeywords{blockchain; navigation; GeoHash }

\end{englishabstract}
