%%==================================================
%% abstract.tex for BIT Master Thesis
%% modified by yang yating
%% version: 0.1
%% last update: Dec 25th, 2016
%%==================================================

\begin{abstract}

车载自组网是在交通环境参与者间构建的开放式网络,可以为用户提供去中心化的数据传输能力。基于车载自组网,可以实现事故预警、辅助驾驶、道路交通信息查询、车间通信和网络接入服务等应用。研发这些应用需要地理信息和交通数据的支持,但信息的垄断会引发不正当牟利和恶性竞争。针对这一问题,本文利用部署在车载自组网上的区块链网络,基于GeoHash矢量地图和以太坊平台,开发了一套出租车调度和导航系统以完成出租车的去中心化调度。首先,本文选用GeoHash作为系统中统一的位置信息表示方法,在系统实现上采用浏览器与智能合约相结合的方式,在智能合约端开发了基于GeoHash的路径导航算法和车辆的区域调度算法,解决了车乘分配时的并发冲突问题。在浏览器端实现车辆和乘客的数据采集和乘车业务完整流程的设计,充分利用了区块链的性质保证车辆信誉数据的安全性、可溯性和在网络内的同步性,并利用GeoHash在地理信息上的计算特性对算法速度进行了优化,并进行了优化后的实验验证工作。最后调节系统的关键参数进行性能优化并进行实验验证,通过真实的地图数据验证了此出租车调度系统的可行性。

\keywords{区块链; 导航; GeoHash }
\end{abstract}

\begin{englishabstract}

   In order to exploit …….
   
\englishkeywords{shape memory properties; polyurethane; textile; synthesis; application}

\end{englishabstract}
