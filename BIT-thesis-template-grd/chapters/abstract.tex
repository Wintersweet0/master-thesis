%%==================================================
%% abstract.tex for BIT Master Thesis
%%==================================================

\begin{abstract}

车载自组网是在交通环境参与者间构建的开放式网络,可以为用户提供去中心化的数据传输能力。基于车载自组网,可以实现事故预警、辅助驾驶、道路交通信息查询、车间通信和网络接入服务等应用。研发这些应用需要地理信息和交通数据的支持,但信息的垄断会引发不正当牟利和恶性竞争。针对这一问题,本文基于GeoHash矢量地图和地理位置区块链平台,开发了一套出租车调度系统,以完成出租车调度的去中心化管理。首先,本文选用GeoHash作为系统中统一的位置信息表示方法。同时,本工作在智能合约中对GeoHash在地理信息上的距离计算进行了优化。之后,本文采用终端与区块链服务端相结合的方式来实现调度系统。其中,在区块链服务端开发了基于GeoHash地图数据的路径规划算法和车辆的区域调度算法,解决了车乘分配时的并发冲突问题,在终端实现车辆和乘客的信息维护和乘车业务的完整流程设计。最后,本工作对距离计算的优化进行了实验验证,并调节路径规划算法的关键参数来保证系统功能的准确性和高效性。最后,进行了模拟道路和真实道路上的系统运行实验,验证了本出租车调度系统的可用性。本文的系统充分利用了GeoHash地理信息和地理位置区块链的性质,保证交通数据的安全性和服务的高效性,证明了把GeoHash地理信息和地理位置区块链融合到车载自组网应用中的可行性。

\keywords{智能交通;区块链;车载自组网;GeoHash}
\end{abstract}

\begin{englishabstract}

VANET is an open network constructed among traffic environment participants, which can provide users with decentralized data transmission capabilities. Based on VANET, applications such as accident warning, assisted driving, road traffic information query, inter-vehicle communication and network access services can be realized. The development of these applications requires the support of geographic information and traffic data, but the monopoly of information will lead to unfair profit-making and vicious competition. In response to this problem, this paper develops a taxi dispatching system based on the GeoHash vector map and geographic location blockchain platform to complete the decentralized management of taxi dispatching. First of all, this paper selects GeoHash as the unified location information representation method in the system. At the same time, this work optimizes the distance calculation of GeoHash on geographic information in the smart contract. After that, this paper adopts the combination of terminal and blockchain server to realize the scheduling system. Among them, the path planning algorithm based on GeoHash map data and the regional scheduling algorithm of vehicles are developed on the blockchain server, which solves the problem of concurrency conflicts in the allocation of vehicles and passengers, and realizes complete process design of vehicle and passenger information maintenance and ride-hailing business at the terminal. Finally, the optimization of the distance calculation is experimentally verified in this work, and the key parameters of the path planning algorithm are adjusted to ensure the accuracy and efficiency of the system function. Finally, the system operation experiments on simulated roads and real roads are carried out to verify the availability of the taxi dispatching system. The system in this paper makes full use of the properties of GeoHash geographic information and geographic location blockchain to ensure the security of traffic data and the efficiency of services, and proves the feasibility of integrating GeoHash geographic information and geographic location blockchain into VANET applications.

\englishkeywords{Intelligent Transportation; Blockchain; VANET; GeoHash}

\end{englishabstract}
